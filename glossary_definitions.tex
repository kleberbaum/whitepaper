\newglossaryentry{neuralnetwork}{
  name={neural network},
  description={Künstliches neuronales Netz aus miteinander verbundenen Einheiten („Neuronen“), das durch Training Muster erkennen kann}
}
\newglossaryentry{weights}{
  name={weight},
  description={Veränderbarer Wert an den Verbindungen zwischen Neuronen, der beim Training angepasst wird}
}
\newglossaryentry{llm}{
  name={LLM},
  description={Large Language Model, ein großes Sprachmodell}
}
\newglossaryentry{token}{
  name={Token},
  description={Kleinste Verarbeitungseinheit eines Sprachmodells, z.B. Wort oder Symbol}
}
\newglossaryentry{bias}{
  name={Bias},
  description={Systematische Voreingenommenheit eines Modells}
}
\newglossaryentry{alignment}{
  name={Alignment},
  description={Anpassung/Ausrichtung eines Sprachmodells an Ziele, Werte oder Rollen}
}
\newglossaryentry{sampling}{
  name={Sampling},
  description={Prozess zur Auswahl des nächsten Tokens}
}
\newglossaryentry{temperatur}{
  name={Temperatur},
  description={Steuerung der Zufälligkeit (Kreativität) bei der Textgenerierung}
}
\newglossaryentry{top-p}{
  name={Top-p},
  description={Nucleus Sampling: Auswahl aus Tokens bis zu einer Gesamtwahrscheinlichkeit p}
}
\newglossaryentry{top-k}{
  name={Top-k},
  description={Sampling aus den k wahrscheinlichsten Tokens}
}
\newglossaryentry{min-p}{
  name={Min-p},
  description={Nur Tokens mit Mindestwahrscheinlichkeit werden ausgewählt}
}
\newglossaryentry{frequency-penalty}{
  name={frequency-penalty},
  description={Bestrafung von Wiederholungen bereits erzeugter Tokens}
}
\newglossaryentry{presence-penalty}{
  name={presence-penalty},
  description={Bestrafung bereits genutzter Tokens im Kontext}
}
\newglossaryentry{max-tokens}{
  name={max tokens},
  description={Maximale Tokenanzahl pro Antwort}
}
\newglossaryentry{context}{
  name={Context},
  description={Bisheriger Text/Chat-Verlauf als Kontext für das Modell}
}
\newglossaryentry{inferencing}{
  name={Inferencing},
  description={Prozess des Schlussfolgerns/Antwortens eines Modells}
}
\newglossaryentry{inferencer}{
  name={Inferencer},
  description={Das Programm/System, das mit dem Modell arbeitet}
}
\newglossaryentry{gguf}{
  name={GGUF},
  description={Effizientes Speicherformat für LLMs}
}
\newglossaryentry{q6k}{
  name={Q6\_K},
  description={6-Bit gewichtsquantisiertes Modell: für effiziente Anwendung}
}
\newglossaryentry{eca}{
  name={ECA},
  description={Embodied Conversational Agent}
}
\newglossaryentry{tavernai}{
  name={Silly Tavern},
  description={Webbasierte ECA-Plattfor}
}
\newglossaryentry{ce}{
  name={CE},
  description={Character Expressions}
}